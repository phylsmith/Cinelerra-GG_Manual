\chapter{FAQ, Known Problems and Workarounds}%
\label{cha:faq_problems_workarounds}
\index{workarounds}

Some workarounds for issues and a few known problems that have not yet been fixed, are described here.

\paragraph{Workaround for access to modifiable text files when using AppImage.}  This includes
such files as FFmpeg video or audio files, fonts that you want to add for usage in the Title
plugin, icons that you wish to modify, standalone programs, and any other text file for customization.
To get access, first extract the files from the AppImage, find the files to add, edit, or modify, and then run the extracted binary following the steps shown here.
\begin{itemize}
	\item  /path-to-appimage/CinGG-20220131-x86\_64.AppImage -{}-appimage-extract
	\item  edit or add the files you want to modify in the \textit{bin} or \textit{lib} subdirecctory
	\item  /path-to-appimage/squashfs-root/usr/bin/cin
\end{itemize}
For more detailed information on working with AppImages, see \ref{sub:managing_appimage}.

\paragraph{Workaround for the \textit{alpha channel not working as intended when using an EDL} inserted as a reference or when an EDL is nested.}
Using the Projector on the EDL within a Master Project follow these next steps.
\begin{itemize}
	\item Open your EDL with \textit{Open EDL} by using the MMB (Middle Mouse Button) on the Nested edit on its track.
	\item In this nested EDL, insert a new Video track below the last video track.
	\item Add the Alpha plugin to the new Video track for the entire length of the nested EDL.
	\item Change the value of the Alpha to 0.00. The Alpha Plugin may be disabled or left enabled.
	\item Close the EDL.
\end{itemize} This solves the alpha channel not working as expected and the Master Project now looks correct.

\paragraph{Workaround for using a transition, like Dissolve, between 2 edits with the plugin Motion51.}
In this case, the Motion51 plugin and the Dissolve transition are not seen during the transition phase
and is stopped at the cut point between 2 edits.
The left edit's plugin should be in effect until the end of that clip and should include the
dissolve.  Instead, there is a jump inside the dissolve, as if the plugin does not exist. There are
2 possible workarounds.  For example, extend the Motion51 plugin to cover both clips rather than
having a separate plugin for each clip.  Or another workaround is to use two tracks and have the 
plugin cover the cut point and use a Fade auto instead of the Dissolve transition.

\paragraph{The Fade auto in certain circumstances produces an unwanted black flash.}
When you insert an effect on an empty track the fade fades as if it were on a black background;
it interprets the alpha as black where there is transparency, which is what produces the fade.
So the problem in the case of text is that the letters appear as a black flash briefly in the first
few frames, then goes back to transparent, then the fade starts normally with the white color. 
To avoid the unwanted effect produced by the fade over black (because there is no content) leave
the fade in/out of the effect at 0 and use the fade line to perform the intended results to
avoid the problem with the alpha channel. The same problem occurs when using transparent PNG
images with fade video transitions and can be resolved in the same manner.
However this workaround can be quite tedious and complicated if you want to fade many snippets of text
and transparent PNGs on the same track, with always the same video transition filter set to the same 
duration. 

Just a side note here. Fade in/out in the Title plugin works fine if there is a clip/image above 
the plugin in the same track without an alpha channel but not if there is no clip/image.
And the Transition effect works fine when there is a transition between two clips/images without
an alpha channel.

\paragraph{Do not change Color Model Format in session just before loading backup.}
Under certain circumstance if you change the Color Model in Settings->Format, quit the session,
and then "Load backup" when you start again, it will crash.  Workaround is to avoid this scenario.

\paragraph{Audio tracks align with no video track, and in working with Mixers.}

The use of \texttt{Open Mixers} always requires the existence of a video track. See \nameref{ssub:but_use_only_first_audio} We cannot act on audio tracks only, for example to align them. A workaround to get the audio tracks aligned would be:

\begin{enumerate}
	\item You create/open the \textit{Mixer Viewer} with \texttt{Open Mixers} in Resources window for the Videos with Audio. It will create, in the timeline, one Video track and two Audio tracks for each Media (Video/Audio), and its Mixer Viewer.
	\item Add a new Video track (\texttt{Shift+T} shortcut) and two Audio tracks (\texttt{t} shortcut) for your audio (Left+Right channels?) without video: you are adding an empty Video track because \texttt{Align Mixers} feature needed a video track.
	\item Move the Video track down until it is on the top of the two Audio
	tracks.
	\item Insert your Audio into the two Audio tracks in the usual way.
	\item Open/Create a Mixer Viewer from \texttt{Menu $\rightarrow$ Window $\rightarrow$ Mixer\dots $\rightarrow$ Mixer Viewer} (\texttt{Shift+M}).
	\item Assign that Mixer Viewer (a white frame is shown when selected) to the one Video tracks (empty) and two Audio tracks, of the \#2 point, pressing the arrow icons (\texttt{Mixer}) you find on the Patchbay on the left of the Main window. When assigned, you  must to see the 3 arrow (for 1video + 2Audio) are pointing up.
	\item Now, you can use \texttt{Align Mixers} feature as usual. See \nameref{sec:audio_video_sync}
\end{enumerate}

