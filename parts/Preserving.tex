\chapter{Overview}%
\label{cha:camcorders}

This document describes the methods used to achieve a "backup type" or intermediate codec
for non-edited source video camcorder formats, compatible with and viewable on Bluray hardware players. After you have
moved your camcorder media onto a computer disk, you can with some extra effort, transfer them for safekeeping and
longer term storage on simple viewable Blu-ray video discs. 

Transferring camcorder media to digital files on a computer, and next to optical Blu-ray video media is especially
important for legacy tape-based camcorders, because tape will degrade over time. You can do this using a combination of
free tools on Linux, generic and independent of any NLE. 

In this case we use and keep interlaced video as on the source or digitized video (PAL 50i, 25 fps) for both DV and HDV
formats.  As confirmed by the ffmpeg output below, DV uses 25 Mbps data rate (DV25), is Standard Definition (SD),
resolution 720{\texttimes}576, 8 bits color depth 4:2:0, display aspect ratio (DAR) 4:3 while HDV is High Definition HD video,
resolution 1440{\texttimes}1080, DAR 16:9, 8 bits color depth 4:2:0. 

We utilize or keep relative low MPEG-2 video compression at the same bit-rate 25 Mbps as the original video recorded
onto the source tapes. That is about 13 GB per hour video and audio in total, and about 3 hours playtime with 40 GB
BD-video size (of max 46.5 GB) on each 50 GB BD-R DL Blu-ray disc. Audio is transcoded to uncompressed LPCM compatible
with BD-video. 

In this how-to we combine \textbf{ffmpeg}, \textbf{tsMuxeR} and \textbf{xorriso} to create the MTS-video stream, to
create the Blu-ray ISO image, and to burn the image to Blu-ray video disc. The benefit of using tsMuxer here, is its
support of UDF version 2.50 as required by the Blu-ray video standard. Listed next are the possible camcorder video formats on Blu-ray video discs.

\begin{itemize}
\item \textbf{DV} (1995) transfer to SD-blu-ray video (re-encoding video to mpeg2, re-encoding DV LPCM audio to Blu-ray LPCM) 
\item \textbf{HDV} (2004) transfer to blu-ray video (copying mpeg2 video, re-encoding MP2 audio to Blu-ray LPCM or AC3) 
\item \textbf{AVCHD} (2006-current) transfer to Blu-ray video (copying H.264 video and AC3 audio)
\end{itemize}

An alternative method as described in the CinGG manual, is to use \textit{mkudffs} and \textit{bdwrite} packaged with CinGG, to create the udfs image and burn it to Blu-ray disc using \textit{growisofs} or \textit{dd}. \\
{\small \url{https://cinelerra-gg.org/download/CinelerraGG\_Manual/Creating\_Blu\_ray\_Without\_Re.html}}

There are other methods described on the internet and there are alternatives to the software programs used here that work as well. 

\chapter{Video recording formats}%
\label{cha:recording}

A description of the most common video recording formats for consumer and prosumer equipment; camcorders, analog video
cassette recorders (VCR decks), digital video hard disc recorders, and optical disc recorders (DVD, Blu-ray).

\section{\textbf{Analog video} (legacy) onto magnetic tape cassettes}%
\label{sec:analog}
\settocdepth{subsection}

\subsection{\textbf{VHS (1976), VHS-C (1982), and Video8 (1984)}}%
\label{sub:vhs1976}

%
% WARNING - the next line will change it for every list for rest of document
%
%\setlist{noitemsep}
\begin{itemize}[noitemsep]
\item Composite Audio/Video is encoded on one channel with maximum horizontal resolution 240 (250) dots (or visually
resolvable vertical picture lines) for VHS/VHS-C and 280 dots for Video8 (2.7MHz video bandwidth).
\item In comparision, horizontal resolution bandwidth for PAL broadcast: 5.0-5.5 MHz, NTSC: 4.2 MHz.
\item It is the horizontal resolution that makes the big difference in picture quality since all video formats have the
same number of vertical resolution (or visible horizontal scan lines); always 576 for PAL/SECAM and 486 for NTSC.
\item RCA (phono) connectors: yellow for composite video, red and white for left and right audio signals.
\item 21-pin SCART Euro connector with optional RCA adapter.
\item Magnetic tape media in full size VHS (1/2") cassettes, VHS-C (1/4") compact and 8\ mm cassettes.
\item Cassette adapters to playback compact VHS-C cassettes on full size VHS decks/players.
\end{itemize}

\subsection{SVHS (Super VHS, 1987), SVHS-C (-Compact, 1987) and Hi8 (1989)}%
\label{sub:svhs1987}

\begin{itemize}[noitemsep]
\item S-Video (separate video, Y/C) encodes video \href{https://en.wikipedia.org/wiki/Luma\_(video)}{luma}
(brightness/contrast/black\&white) and
\href{https://en.wikipedia.org/wiki/Chrominance}{chrom}\href{https://en.wikipedia.org/wiki/Chrominance}{a} (color) on
two separate channels, achieving higher image quality and amount of picture details over composite video.
\item Max visible horizontal resolution is 400 dots for SVHS/SVHS-C and 440 dots for Hi8 systems on high-end VCR
machines (4.5 MHz decks). In practice consumer/prosumer camcorder recordings quality vary from about 330-350 dots
horizontal resolution (3.5MHz), which is equivalent to analog TV set quality.
\item 4-pin mini-DIN S-video connector for Y/C video, red and white RCA line connectors for left and right audio;
optional SCART Euro adapter with 4-pin video and RCA audio connectors.
\item In comparision, analog Component video (three component YPbPr) has the highest color resolution and is encoded
over three channels with green, blue and red RCA connectors (consumer) or with BNC connectors (prosumer), additional red
and white phono line connectors for audio.
\end{itemize}

\section{\textbf{Digital video}}%
\label{sec:digital}

\subsection{\textbf{DV (1995), Digital8 (1999)}}%
\label{sub:dv1995}

\begin{itemize}[noitemsep]
\item DV refers to a family of codecs and tape formats used for storing digital video, especially on MiniDV cassettes, the most popular tape format, or on Digital8 cassettes using a DV codec. 
\item DV codecs are still sometimes used when dealing with legacy standard definition SD video:
 \newline\hspace*{.5cm}PAL:  625i50, 720 {\texttimes} 576 frame size, 25 fps, 8 bit 4:2:0 subsampling
 \newline\hspace*{.5cm}NTSC: 525i60, 720 {\texttimes} 480 frame size, 30 fps, 8-bit 4:1:1 subsampling
\item The same frame size is used for 4:3 and 16:9 frame aspect ratios, resulting in different pixel aspect ratios for
fullscreen and anamorphic widescreen video.
\item DV uses lossy, DCT algorithm intraframe video compression 5:1 on a frame-by-frame basis.
\item Audio almost exclusively is stored as 16-bit LPCM at 48 kHz sampling rate, 1.5 Mbit/s stereo.
\item Data rate is about 25 Mbit/s for video, thus DV25, and an additional 1.5 Mbit/s for PCM audio. 
\item Total stream is 29 Mbps, or 3.6 MiB/s. The information output is approximately 200 MiB per minute, 12 GiB or 13 GB
per hour playtime.
\item DV camcorders and decks have IEEE 1394 (FireWire, i.LINK) ports for digital video transfer. 
\item \textbf{DV50} (Digital-S or D-9 and DVCPRO50) is a 50 Mbit/s variant of DV with improved 4:2:2 chroma subsampling and 3.3:1 compression.  This in effect cuts total record time of any given storage medium in half compared to DV25. 
\item DV was strongly associated with the transition from analog to digital desktop video production.
\item Digital8 equipment recorded in DV format only, but usually could playback Video8 and Hi8 tapes as well, and were
also capable to do analog to digital video conversion during playback. 
\item Most (but not all) modern digital camcorders provided an analog-to-digital
"passthrough" capability. This feature fed an analog input signal (usually via S-Video,
sometimes RCA) into the camcorder and output a standard DV signal. The camcorder converts the analog signal on-the-fly
using a hardware encoder. The input device could be your old analog camcorder (playing 8\ mm or Hi8 tapes), a VHS/VHS-C
video player, to convert analog to digital output via Firewire \textbf{dv} file extension.
\end{itemize}

\subsection{\textbf{DVD-Video} (1996)}%
\label{sub:DVD-Video}

DVD-Video discs are intended for full-length movies and offer a range of features including the following:
\begin{itemize}[noitemsep]
\item Playing time: a nominal 133 minutes playing time for DVD-5 or each side of a DVD-10 and 240 minutes for DVD-9 using opposite track path format. In practice playing times are often reduced in favour of improved quality.
\item Video encoding: MPEG-2 (\href{mailto:MP@ML}{MP@ML}) or MPEG-1.
\item Audio Quality and Languages: Dolby Digital, DTS, MPEG-2 or Linear PCM audio for up to 5.1 channel surround sound.
\end{itemize}

\begin{itemize}
\item To record digital video, DVD-Video uses \href{https://en.wikipedia.org/wiki/H.262/MPEG-2\_Part\_2}{H.262/MPEG-2 Part
2} compression at up to 9.8~Mbit/s, maximum of 10.08 Mbit/s can be split amongst audio and video. DVD-Video supports
video with a \href{https://en.wikipedia.org/wiki/Color\_depth}{bit depth} of 8 bits per color, encoded as
\href{https://en.wikipedia.org/wiki/YCbCr}{YCbCr} with 4:2:0
\href{https://en.wikipedia.org/wiki/Chroma\_subsampling}{chroma subsampling}. The H.262/MPEG-2 Part 2 format supports
both interlaced and progressive scan content.
\item As for lines of horizontal resolution, DVD has about 500. In analog output signal terms, typical luma frequency
response maintains 53/145 full amplitude to between 5.0 - 5.5 MHz. This is below the 6.75 MHz native frequency of the
MPEG-2 digital signal. Chroma frequency response is one-half that of luma.
\end{itemize}

The following formats are allowed for H.262/MPEG-2 Part 2 video:
\begin{itemize}
\item At a display rate of 25 frames per second, interlaced or progressive scan (commonly used in regions with 50 Hz
image scanning frequency, compatible with analog 625-line PAL/SECAM):
 \begin{adjustwidth}{.5cm}{0cm}
 720 {\texttimes} 576 pixels (D-1 resolution, 4:3 fullscreen or 16:9 anamorphic widescreen aspect ratio)
 \end{adjustwidth}
\item  At a display rate of 29.97 frames per second, interlaced or progressive scan (commonly used in regions with 60 Hz
image scanning frequency, compatible with analog 525-line NTSC):
 \begin{adjustwidth}{.5cm}{0cm}
 720 {\texttimes} 480 pixels (D-1 resolution, 4:3 or 16:9)
 \end{adjustwidth}
\end{itemize}

The official allowed formats for the audio tracks on a DVD-Video are:
 \begin{itemize}
 \item PCM: 48 kHz or 96 kHz sampling rate, 16 bit or 24 bit Linear PCM, 2 to 6 channels, up to 6,144 kbit/s; N. B.
16-bit 48 kHz 8 channel PCM is allowed by the DVD-Video specification but is not well supported by authoring
applications or players.
 \item  AC-3: 48 kHz sampling rate, 1 to 5.1 (6) channels, up to 448 kbit/s.
 \item  MP2: 48 kHz sampling rate, 1 to 7.1 channels, up to 912 kbit/s.
 \item  DTS: 48 kHz or 96 kHz sampling rate; channel layouts = 2.0, 2.1, 5.0, 5.1, 6.1; bitrates for 2.0  and 2.1 =
377.25 and 503.25 kbit/s, bitrates for 5.x and 6.1 = 754.5 and 1509.75 kbit/s.
\item File system:
\begin{itemize}[noitemsep]
\item Almost all DVD-Video discs use the UDF bridge format, which is a combination of the DVD MicroUDF (a subset of UDF 1.02) and ISO 9660 file systems.
\item  The UDF bridge format provides backwards compatibility for operating systems that
support only ISO 9660. 
\item Most DVD players read the UDF filesystem from a DVD-Video disc and ignore the ISO9660 filesystem.  
\end{itemize}
\end{itemize}

\subsection{\textbf{HDV (2003)}}%
\label{sub:HDV2003}

\begin{itemize}[noitemsep]
\item HDV was a successor format to DV with an updated video codec, and used the same MiniDV cassette format, optional
specifically for HDV recording on tape with reduced drop-out rate.
\item Some manufacturers offered on (larger) camera hard disk recording units capable of recording HDV both onto tape
and/or onto file-based media via FireWire connection. 
\item HDV camcorders can typically switch between HDV/MP2 and SD-DV/PCM recording modes.
\item HDV 1080i with horizontal resolution 1440 is twice the 720 resolution of DV and DVD, and the perceived
sharpness with HDV is much higher when scaled up to full HD (TV) resolution.
\end{itemize}

\begin{tabular}{ll}
------------------- \\
{\textbullet}\textbf{HDV 1080i} specification (HD2): & \textbf{HDV 720p} specification (HD1): \\
{\textbullet} 1080/ 50i (PAL), 1080/ 60i (NTSC) &  720/25p, 720/50p,720/30p, and 720/60p \\
{\textbullet} 1440 {\texttimes} 1080, anamorphic pixels 4:3=1.33 & 1280 x 720, Pixel aspect ratio 1.0  \\ 
{\textbullet} 16:9 Display aspect ratio & 16:9 Display aspect ratio \\
\hspace*{.5cm} - 720{\texttimes}480/ 60i  (SD-DV 4:3 og 16:9) \\
\hspace*{.5cm} - 720{\texttimes}576/ 50i  (SD-DV 4:3 og 16:9) \\
{\textbullet} MPEG-2 interframe GOP (\href{mailto:MP@H-14}{MP@H-14}) & MPEG-2 (\href{mailto:MP@H-14/HL}{MP@H-14/HL}) \\
{\textbullet} 8 bit color depth 4:2:0 subsampling & 8 bit color depth 4:2:0 subsampling \\
{\textbullet} 25 Mbps data rate after compression (20:1) & 19.4 Mbps data rate after compression \\
{\textbullet} MP2/MPEG-1 Audio Layer II & MP2/MPEG-1 Audio Layer II \\ 
{\textbullet} Stereo (2-ch), 384 kbps & Stereo (2-ch), 384 kbps \\
{\textbullet} A/V out: HDMI, Component, S-video/RCA\ &  A/V out: HDMI, Component, S-video/RCA \\
{\textbullet} IEEE 1394 (MPEG-2-TS) stream interface & IEEE 1394 (MPEG-2-TS) stream interface \\
{\textbullet} M2T file extension & M2T file extension
\end{tabular}

\subsection{\textbf{AVCHD (2006-current)}}%
\label{sub:AVCHD}

\begin{itemize}
\item AVCHD stands for \textit{Advanced Video Coding High Definition}, in particular, \textit{MPEG-4 Part 10:
AVC/H.264}. The intent of the H.264/AVC project was to create a standard capable of providing good video quality at
substantially lower bit rates than previous standards (i.e., half or less the bit rate of MPEG-2, H.263, or MPEG-4 Part
2). \\ 
\item AVCHD is a file based format for digital camcorders to record and playback 1080i and 720p HD-video onto certain
random access media. Memory cards, thumb drives, and HDDs use the FAT file system. This highly compressed video allows for
recording long videos in high definition.\\
\item AVCHD was originally a simplified version of the Blu-ray standard to enable DVD and BD-based camcorders.
Recordable optical discs use UDF or ISO9660 derived from the Blu-ray disc specification. For example, it utilizes a
legacy 8.3 file naming system while Blu-ray disc uses long filenames. \\
\item MTS file extension changes to M2TS when MTS recorded data is transferred to the computer for storing the video in
a Blu-ray disc. \\
\item Playback is possible on an AVCHD-compatible Blu-ray Disc player/recorder, DVD player/recorder, or PlayStation 3, 
the 8 cm DVDs (discs recorded in AVCHD) you have recorded or DVDs (discs recorded in AVCHD) and Blu-ray Discs created
by importing videos to a PC or Blu-ray Disc player. \\
\item Today AVCHD is the most popular camcorder format for consumers and prosumers, and is also available as AVCHD
Progressive/PCM models for professional use.\\
\item AVCHD supports both \textit{AVCHD-SD Standard Definition} and \textit{AVCHD 1080i High Definition} interlaced
video, while AVCHD 1080i is available with most AVCHD camcorders. AVCHD supports 720-line progressive recording mode at
frame rates of 24 and 60 frames/s for 60 Hz models and 50 frames/s for 50 Hz models.\\
\item Also, 3D (MVC format) and 1080/60p(1080/50p) video formats were added as an extension to the AVCHD format to
create the \textit{AVCHD Ver. 2.0, 2011} (AVCHD 3D, AVCHD Progressive) format. Compatibility in AVCHD Ver. 2.0 format compliant devices is secured by being standardized as the AVCHD format.\\
\end{itemize}
--------------------------------------

\begin{tabular}{ll}
 \textbf{8\ cm DVD/Built-in/SD Memory/M-Stick: } & \textbf{Built-in media/SD Memory/Memory Stick:} \\

 {\textbullet} 1920{\texttimes}1080/ 60i, 50i, 24p &  1920{\texttimes}1080/ 60p, 50p \\ 
 {\textbullet} 1440{\texttimes}1080/ 60i, 50i, 24p & 1440{\texttimes}1080/ 60p, 50p \\
 {\textbullet} 1280{\texttimes}720/ 60i, 50i, 24p & 1280{\texttimes}720/ 60p, 50p \\
 {\textbullet} 720{\texttimes}480/ 60i  (SD 4:3 og 16:9)  \\
 {\textbullet} 720{\texttimes}576/ 50i  (SD 4:3 og 16:9) \\ 
 {\textbullet} 16:9 Display aspect ratio & 16:9 Display aspect ratio \\
 {\textbullet} MPEG-4 Part 10: AVC / H.264 & MPEG-4 Part  10: AVC / H.264 \\
 {\textbullet} 8 bit color depth 4:2:0 subsampling & 8 bit color depth 4:2:0 subsampling \\
 {\textbullet} {\textless}= 24 Mbps data rate & {\textless}= 28 Mbps data rate \\
 {\textbullet} {\textless}= 18 Mbps data rate for DVD   \\
 {\textbullet} Dolby Digital AC-3\ \ , 64-640 kbps &  Dolby Digital AC-3, 64-640 kbps \\ 
 \hspace*{.3cm} 2 ch stereo and 5.1  (5-ch + subw.) surround & \hspace*{.3cm} 1$\sim$ 5.1 channels \\
 {\textbullet} Linear PCM,  1.5 Mbps (2 ch) & Linear PCM,  1.5 Mbps (2 ch)  \\
  \hspace*{.3cm}1$\sim$ 7.1 channels & \hspace*{.3cm} 1$\sim$ 7.1 channels \\
 {\textbullet} A/V out: HDMI, Component, S-video/RCA &  A/V out: HDMI, Component, S-video/RCA \\
 {\textbullet} MPEG-2 Transport Stream & MPEG-2 Transport Stream \\
 {\textbullet} MTS file extension\ \ (recorded) & MTS file extension (recorded) \\
\end{tabular}

\subsection{\textbf{Blu-ray BD-Video (2006), UHD-video (2016)}}%
\label{sub:Blu-ray}

All standard DVDs will play on existing Blu-ray players, making the switch to Blu-ray much easier than the switch
from VHS to DVD. Ultra HD Blu-ray is the latest version available, supporting 4K resolution content. Historically,
Blu-ray Disc allows video with a color (bit) depth of 8-bits per color YCbCr with 4:2:0 chroma subsampling (colors
compressed to 25\% of uncompressed), which means 256 possible values for red, green and blue; 16.7 million colors in
total. Ultra-HD Blu Ray is 10-bit 4:2:0 to reduce color banding, giving 1024 values for RGB, 1.0 billion colors in
total, or 64x more than 8-bit.

BDMV Video encoding: 
\begin{itemize}[noitemsep]
\item UDF2.5 as file system (BD-R/RE, 2005)
\item M2TS file extension
\item H.262/MPEG-2 Part 2 (\href{mailto:MP@HL}{MP@HL}, \href{mailto:MP@ML}{MP@ML})
\item H.264/MPEG-4 AVC (\href{mailto:HP@L4.1}{HP@4.1}, \href{mailto:MP@L4.1}{MP@4.1})
\item SMPTE VC-1 (\href{mailto:AP@L3}{AP@L3})
\item H.265/H.265/MPEG-H Part 2 (HEVC) (only Ultra HD Blu-ray on High-density optical disc)
\end{itemize}

BD Video movies have a maximum data transfer rate of 54 Mbit/s, a maximum AV bitrate of 48 Mbit/s (for both audio
and video data), and a maximum video bit rate of 40 Mbit/s., BD disc capacities, each with its own data rate: 
\begin{itemize}[noitemsep]
\item 25 GB (BD-R SL)
\item 50, 66 GB (BD-R DL) at 72 or 92 Mbit/s (50 GB BD-R DL 4x read speed supports UHD-video)
\item 100 GB (TL), 128 GB (QL) at 92, 123, or 144 Mbit/s (BD-R XL)
\end{itemize}

Supported video formats (shortened):

\begin{tabular}{ l c c}
\textbf{Format} & \textbf{Resolution and frame rate} & \textbf{Display aspect ratio }\\

4K UHD & 3840x2160 60p, 50p & 16:9 \\
         & 3840x2160 25p, 24p & 16:9 \\
\\
HD & 1920x1080 60p, 50p & 16:9 \\
   & 1920x1080 25p, 24p & 16:9 \\
   & 1920x1080 29.97i, 25i & 16:9 \\
   & 1440x1080 29.97i, 25i & 16:9\\
\\
HD & 1440x1080 24p & 16:9\\
   & 1280x720 59.94p, 50p 1 & 16:9 \\
   & 1280x720 24p & 16:9 \\
\\
SD & 720x480 29.97i, 25i & 4:3 or 16:9\\
\end{tabular}

\chapter{How to digitize and capture analog video to digital SD video format with A/D video conversion}%
\label{cha:how_to}

\section{Methods and workflow}%
\label{sec:methods}

There are a number of ways to get analog source recorded video digitized and captured into the computer. 
\begin{itemize}
\item One way is to use the passthrough feature of an available miniDV camcorder or the digital output of a Sony Digital 8
camcorder to convert the Hi8 source to DV, which is then sent via firewire to the computer.
\item Another alternative might be a Linux supported external capture card to USB3 (often MPEG-4), or find a higher-end
internal PCIe analog video capture card to store uncompressed, 422 or DV video. 
\item A third way is to use a standalone analog to digital A/D converter and capture device. This might be to DV HDD recorder
or optional to a high-end analog/SDI and a SSD 422 codec (i.e ProRes) recorder.
\end{itemize}
\hspace*{.9cm} An example workflow and setup for this third way:

\small{Analog Hi8 \hspace*{.15cm}            S-Video  \hspace*{.8cm}           S-Video} \\
       tape player ----------> TBC --------------> A/D-conv. \\
\hspace*{8cm}                                            $\backslash$ \_ DV/HDV HDD \hspace*{.5cm} Firewire \hspace*{.5cm} PC \\
\hspace*{8.3cm}                                                       \_ rec/player  ------------------------> DV/M2T \\
\hspace*{6.4cm}                                              Firewire \hspace*{.1cm}/ \\
\hspace*{3.5cm}                              HDV tape player ----------->

\begin{itemize}
\item A fourth, interesting way as discussed on the CinGG mailing list (yet to be tested), is to combine a HDMI-USB3 capture dongle with an Analog Video to Digital HDMI miniConverter (ADC). The analog video player output (via TBC) connects to the actual input connectors (S-video/RCA or possibly Component) of the adapter.  Reference the following from/to: \\
{\small \url{https://lists.cinelerra-gg.org/pipermail/cin/2021-October/003960.html}} \\
{\small \url{https://lists.cinelerra-gg.org/pipermail/cin/2021-October/003970.html}}

\end{itemize}

{\small \texttt{S-Video Out -> A/D video converter -> HDMI/USB3-> HDMI/USB3 capture card}}

\section{Equipment and setup}%
\label{sec:equipment}

\begin{itemize}
\item The first required is a tape player device (VCR deck or camcorder) to playback the analog video tape cassette formats,
either VHS/ SVHS, VHS-C/ SVHS-C or Video8/ Hi8. Use a video head cleaning tape before use. Wind and rewind the tape
cassette before playback in Edit mode "ON" to minimize picture deterioration. \\
\item A second recommended device for all capture methods is a Time Base Corrector (TBC).  Connected to the analog output of
a VCR, a good TBC can make wonder and remove most or all of the flagging, shaky pictures, wavy lines, and other time
base problems from a videotape, to get a clearer and more steady picture. The best is a standalone, fullframe
high-resolution TBC of broadcast quality; minimum is a line-based TBC built in high-end VCR decks. \\
\item A third, optional feature before digitizing is a noise filter to reduce/soften noise and grain (snow) from bad VHS tape
recordings, or also when video that is over-processed using sharpness controls and enhancers. Noise filter adjustment
may be part of the TBC setup and preview, or fixed built in the playback VCR.  Optional a standalone Video
corrector/processor if available can be used, where also optional brightness, contrast and color saturation/correction
can be adjusted.
\end{itemize}

\section{Example articles and references}%
\label{sec:example}

\begin{itemize}
\item Digitizing Analog Video through a Digital Camcorder (Michael Steil, 2022) \\
{\small \url{https://www.pagetable.com/?p=1697}} \\
\item How to Get the Best Images Digitizing SD Video (TBC/ProRes, Larry Jordan, 2021) \\
{\small \url{https://larryjordan.com/articles/how-to-get-the-best-images-digitizing-sd-video/}} \\
\item Digitize analog cassettes (VHS or 8\ mm) with Linux (USB/H.264, Corinne HENIN, 2021) \\
{\small \url{https://www.arsouyes.org/en/blog/2021/2021-05-17\_Numerisation\_VHS}} \\
\item The Digitization of VHS Videotapes -- Technical Bulletin 31 -- Canada.ca (Joe Iraci, 2020) \\
{\small \url{https://www.canada.ca/en/conservation-institute/services/conservation-preservation-publications/technical-bulletins/digitization-vhs-video-tapes.html}} \\
\item Hi8 to DVD Workflow in Linux (Eric Olson, Renomath (2012) \\
{\small \url{https://renomath.org/video/linux/hi8/}} \\
\item Transfer VHS/DVD Media or Video8/Hi8 Tapes into CINELERRA-GG (chpt. 13.4 manual) \\
{\small \url{https://cinelerra-gg.org/download/CinelerraGG\_Manual/Transfer\_VHS\_DVD\_Media\_or\_V.html}}
\end{itemize}

\chapter{\textbf{DV files converted to Blu-ray compliant MPEG-2/LPCM and authored to SD BD-Video}}%
\label{cha:dv_files}

\textbf{Step 1}

\begin{itemize}
\item Prepare a 40 GB continuously DV input file by joining and concatenating the actual dv clips in order. Each
digitized Hi8 video tape was automatically split and recorded as a numbered series of 2.0 GB dv files each. \\
\item Naming convention used, for example, for Tape \#1 with clips in order: \texttt{dv01, dv01-01, dv01\_02, dv01\_03},
etc. For the full example at the end of this article, the concatenated \texttt{DV41\_02+DV42+DV43\_02} contains the clips from on
 \texttt{dv41\_02+all dv42} up to and including \texttt{dv43\_02}.
\end{itemize}

\textbf{Step 2}

\begin{itemize}
\item Encode DV to MPEG-2 video and "re-encode" uncompressed PCM to Blu-ray PCM audio and
remux to interlaced SD-BD-video.mts: 
\begin{lstlisting}[style=sh]
ffmpeg -i DV41_02+DV42+DV43_02.dv -c:v mpeg2video -refs 1 -bf 2 -b:v 25M -maxrate 25M -minrate 25M -bufsize 28M -muxrate 28M -dc 10 -c:a pcm_bluray -mpegts_m2ts_mode 1 -flags +ilme+ildct SD-BD_DV41_02+DV42+DV43_02.mts
\end{lstlisting}
\end{itemize}

\textbf{Step 3}

\begin{itemize}
\item Run \textbf{tsMuxer} (screenshot of tsMuxer usage is displayed at the end of this article). The following
parameters are shown for tsMuxer\_SD\_M2TS: \\
\newline\hspace*{.5cm} Input file: \texttt{SD-BD\_DV41\_02+DV42+DV43\_02.mts} 
\newline\hspace*{.5cm} Tracs:\hspace*{.63cm}\ MPEG-2 video stream and LPCM audio stream 
\newline\hspace*{.5cm} Output:\hspace*{.37cm} \texttt{SD-BD\_DV41\_02+DV42+DV43\_02.iso}
\end{itemize}

\textbf{Step 4}
\begin{itemize}
\item Burn the iso SD-BD-Video image with \textbf{xorriso} on a BD-R DL 50 GB disc:
\begin{lstlisting}[style=sh]
xorriso -as cdrecord -v -sao dev=/dev/sr1 SD-BD_DV41_02+DV42+DV43_02.iso
\end{lstlisting}
\end{itemize}

\chapter{\textbf{HDV.m2t files converted and authored to HD BD-Video}}%
\label{cha:hdvm2t}

\textbf{Step 1}
\begin{itemize}
\item Prepare the source HDV.m2t file using hdv clips as for DV above.
\end{itemize}

\textbf{Step 2}
\begin{itemize}
\item Copying the mpeg2 video, re-encoding MP2 audio to Blu-ray LPCM. If using an ffmpeg version lower than 5.1,
substitute ac3 for pcm\_bluray and add -b:a 384 to override lower default. 

\begin{lstlisting}[style=sh]
ffmpeg -i HDV.m2t -c:v copy -c:a pcm_bluray -mpegts_m2ts_mode 1 HDV.mts 
\end{lstlisting}
\end{itemize}

\textbf{Step 3}
\begin{itemize}
\item Run \textbf{tsMuxer} (screenshot of tsMuxer usage is displayed at the end of this article). 
\newline\hspace*{.5cm} Input file: \texttt{HD-BD\_HDV.mts} 
\newline\hspace*{.5cm} Tracs: \hspace*{.5cm} MPEG-2 video stream and LPCM audio stream 
\newline\hspace*{.5cm} Output: \hspace*{.25cm} \texttt{HD-BD\_HDV.iso}
\end{itemize}

\textbf{Step 4}
\begin{itemize}
\item Burn the iso HD-BD-Video image with \textbf{xorriso} on a BD-R DL 50 GB disc:
\begin{lstlisting}[style=sh]
 xorriso -as cdrecord -v -sao dev=/dev/sr1 HD-BD_HDV.iso
\end{lstlisting}
\end{itemize}

\chapter{\textbf{Full example output}}%
\label{cha:full}

\section{DV files converted to Blu-ray compliant MPEG-2 video and LPCM audio, authored to SD BD-Video and burned to a BD-R
DL}%
\label{sec:DVFILES}

17/01-2023 \\
SD-BD\_DV41\_02+DV42+DV43\_02 \\
-----------------------------------------------

\textbf{Step 1}
\begin{lstlisting}[style=sh]
cd /run/media/terje/Seagate_8TB_back/video/DV

cat dv41_02.dv dv41_03.dv dv41_04.dv dv41_05.dv dv41_06.dv dv41_07.dv dv41_08.dv dv41_09.dv dv42.dv dv42_01.dv dv42_02.dv dv42_03.dv dv42_04.dv  dv42_05.dv dv42_06.dv dv42_07.dv dv42_08.dv dv42_09.dv dv43.dv dv43_01.dv dv43_02.dv > /home/terje/Videoklipp/SD-BD-DV-iso/DV41_02+DV42+DV43_02.dv
\end{lstlisting}

--------------------------------------
\begin{lstlisting}[style=sh]
cd /home/terje/Videoklipp/SD-BD-DV-iso

du -sh DV*
40G     DV41_02+DV42+DV43_02.dv
\end{lstlisting}
------------------------------------------------

\textbf{Step 2}
\begin{lstlisting}[style=sh]
cd /home_lp154/terje/Videoklipp/SD-BD-DV-iso

ffmpeg -i DV41_02+DV42+DV43_02.dv -c:v mpeg2video -refs 1 -bf 2 -b:v 25M -maxrate 25M -minrate 25M -bufsize 28M -muxrate 28M -dc 10 -c:a pcm_bluray -mpegts_m2ts_mode 1 -flags +ilme+ildct SD-BD_DV41_02+DV42+DV43_02.mts

ffmpeg version 5.1.2 Copyright (c) 2000-2022 the FFmpeg developers
   built with gcc 12 (SUSE Linux)
.............................

(NOTE: the @ sign represents the pound sign in the next 21 lines)

Input @0, dv, from 'DV41_02+DV42+DV43_02.dv':
 Metadata:
   timecode        : 00:19:52:24
 Duration: 03:15:53.28, start: 0.000000, bitrate: 28800 kb/s
 Stream @0:0: Video: dvvideo, yuv420p, 720x576 [SAR 16:15 DAR 4:3], 25000 kb/s, 25 fps, 25 tbr, 25 tbn
 Stream @0:1: Audio: pcm_s16le, 48000 Hz, stereo, s16, 1536 kb/s
Stream mapping:
 Stream @0:0 -> @0:0 (dvvideo (native) -> mpeg2video (native))
 Stream @0:1 -> @0:1 (pcm_s16le (native) -> pcm_bluray (native))
Press [q] to stop, [?] for help
[mpeg2video @ 0x558bb9453ac0] Warning vbv_delay will be set to 0xFFFF (=VBR) as the specified vbv buffer is too large for the given bitrate!
Output @0, mpegts, to 'SD-BD_DV41_02+DV42+DV43_02.mts':
 Metadata:
   timecode        : 00:19:52:24
   encoder         : Lavf59.27.100
 Stream @0:0: Video: mpeg2video (Main), yuv420p(bottom coded first (swapped)), 720x576 [SAR 16:15 DAR 4:3], q=2-31,25000 kb/s, 25 fps, 90k tbn
   Metadata:
     encoder        : Lavc59.37.100 mpeg2video
   Side data:
     cpb: bitrate max/min/avg: 25000000/25000000/25000000 buffer size: 28000000 vbv_delay: N/A
 Stream @0:1: Audio: pcm_bluray, 48000 Hz, stereo, s16, 128 kb/s
   Metadata:
     encoder          : Lavc59.37.100 pcm_bluray
frame=293832 fps=373 q=2.5 Lsize=41027202kB time=03:15:53.28 bitrate=28595.8kbits/s speed=14.9x  
video:35867310kB audio:2212922kB subtitle:0kB other streams:0kB global headers:0kB muxing overhead: 7.738845%

--------------------------------------
du -sh *.mts
40G      SD-BD_DV41_02+DV42+DV43_02.mts
\end{lstlisting}
--------------------------------------

\textbf{Step 3}
\begin{itemize}
\item Run \textbf{tsMuxer} (screenshots of tsMuxer usage are displayed at the end of this article). 
\newline\hspace*{.5cm} Input file: \texttt{HD-BD\_HDV.mts} 
\newline\hspace*{.5cm} Tracs: \hspace*{.5cm} MPEG-2 video stream and LPCM audio stream 
\newline\hspace*{.5cm} Output: \hspace*{.2cm} \texttt{HD-BD\_HDV.iso}
\end{itemize}

-------------------------------- 
\begin{lstlisting}[style=sh]
du -sh *.iso
39G     SD-BD_DV41_02+DV42+DV43_02.iso
\end{lstlisting}
--------------------------------------

\textbf{Step 4}
\begin{lstlisting}[style=sh,escapechar=\^]
MediaRange BD-R DL

eject /dev/sr1
eject -t /dev/sr1

umount /dev/sr1
umount: /dev/sr1: not mounted.
--------------------------------------
 
xorriso -devices
xorriso 1.5.4 : RockRidge filesystem manipulator, libburnia project.

Beginning to scan for devices ...
Full drive scan done
--------------------------------------
0  -dev '/dev/sr0' rwrw-- :   'HL-DT-ST' 'BD-RE BH10LS30'
1  -dev '/dev/sr1' rwrw-- :   'ASUS    ' 'BW-16D1X-U'
--------------------------------------

Device og media info for a new BD-R DL (MediaRange):
--------------------------------------
xorrecord -dev=/dev/sr1 -atip
xorriso 1.5.4 : RockRidge filesystem manipulator, libburnia project.

Drive current: -outdev '/dev/sr1'
Media current: BD-R sequential recording
Media status : is blank
Media summary: 0 sessions, 0 data blocks, 0 data, 46.6g free
Device type  : Removable CD-ROM
Vendor_info  : 'ASUS'
Identifikation : 'BW-16D1X-U'
Revision  : 'A105'
Driver flags  : BURNFREE
Supported modes: SAO TAO
Current: BD-R sequential recording
Profile: 0x0043 (BD-RE)
Profile: 0x0042 (BD-R random recording)
Profile: 0x0041 (BD-R sequential recording) (current)
Profile: 0x0040 (BD-ROM)
Profile: 0x002B (DVD+R/DL)
Profile: 0x001B (DVD+R)
Profile: 0x001A (DVD+RW)
Profile: 0x0016 (DVD-R/DL layer jump recording)
Profile: 0x0015 (DVD-R/DL sequential recording)
Profile: 0x0014 (DVD-RW sequential recording)
Profile: 0x0013 (DVD-RW restricted overwrite)
Profile: 0x0012 (DVD-RAM)
Profile: 0x0011 (DVD-R sequential recording)
Profile: 0x0010 (DVD-ROM)
Profile: 0x000A (CD-RW)
Profile: 0x0009 (CD-R)
Profile: 0x0008 (CD-ROM)
Profile: 0x0002 (Removable disk)
Mounted Media: 41h, BD-R sequential recording
Product Id:  CMCMAG/DI6/0
Producer:  CMC Magnetics Corporation
Manufacturer:  'CMCMAG'
Media type:  'DI6'
-------------------------------------- 
\end{lstlisting}

\underline{Burning the 40 GB SD-BD-Video iso image to a BD-R DL disc:}

\begin{lstlisting}[style=sh]
27 min

xorriso -as cdrecord -v -sao dev=/dev/sr1 SD-BD_DV41_02+DV42+DV43_02.iso

xorriso 1.5.4 : RockRidge filesystem manipulator, libburnia project.
Drive current: -outdev '/dev/sr1'
Media current: BD-R sequential recording
Media status : is blank
Media summary: 0 sessions, 0 data blocks, 0 data, 46.6g free
Beginning to write data track.
...........
xorriso : UPDATE : 39219 of 39251 MB written (fifo 71%) [buf 100%]  6.1x.
xorriso : UPDATE : 39245 of 39251 MB written (fifo 80%) [buf 100%]  6.1x.
xorriso : UPDATE : 39251 of 39251 MB written (fifo  0%) [buf 100%]  1.5x.
xorriso : UPDATE : 39251 of 39251 MB written (fifo  0%) [buf 100%]  0.0x.
xorriso : UPDATE : Closing track/session. Working since 1587 seconds
xorriso : UPDATE : Closing track/session. Working since 1588 seconds
..........
xorriso : UPDATE : Closing track/session. Working since 1627 seconds
Writing to '/dev/sr1' completed successfully.

xorriso : NOTE : Re-assessing -outdev '/dev/sr1'
xorriso : NOTE : Disc status unsuitable for writing
Drive current: -outdev '/dev/sr1'
Media current: BD-ROM
Media status : is written , is closed
Media summary: 1 session, 20096864 data blocks, 38.3g data,  0 free
\end{lstlisting}

\begin{figure}[htpb]
\centering
\includegraphics[width=15.132cm,height=9.823cm]{Preserving.png}
\end{figure}
